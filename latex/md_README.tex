Arduino library for S\+PI access to the P\+N532 N\+FC chip in the Italian Electronic Identity Card (C\+IE).

\subsection*{Prerequisites}

This library depends upon the {\bfseries Adafruit\+\_\+\+P\+N532} library which you can obtain from its Git\+Hub Repository.

\href{https://github.com/adafruit/Adafruit-PN532}{\tt https\+://github.\+com/adafruit/\+Adafruit-\/\+P\+N532}

You must install both the Adafruit\+\_\+\+P\+N532 and \hyperlink{classcie__PN532}{cie\+\_\+\+P\+N532} in your Arduino {\itshape libraries} directory. Plese follow the instructions from the Arduino guide.

\href{https://www.arduino.cc/en/Guide/Libraries}{\tt https\+://www.\+arduino.\+cc/en/\+Guide/\+Libraries}

\subsection*{Wiring P\+N532 breakout to the Arduino Uno for S\+PI communication}

At present, just the S\+PI connection is supported. The examples provided will work with this wiring.



This wiring is also described in detail on this page at the Adafruit website.

\href{https://learn.adafruit.com/adafruit-pn532-rfid-nfc/breakout-wiring}{\tt https\+://learn.\+adafruit.\+com/adafruit-\/pn532-\/rfid-\/nfc/breakout-\/wiring}

\subsection*{Getting started}

Create a new arduino project and set it up like this\+: 
\begin{DoxyCode}
1 \{C++\}
2 //Include some libraries
3 #include <Wire.h>
4 #include <SPI.h>
5 #include <cie\_PN532.h>
6 
7 //Use the cie\_PN532 with the typical wiring, as pointed out above
8 cie\_PN532 cie();
9 
10 void setup(void) \{
11   #ifndef ESP8266
12     while (!Serial); // for Leonardo/Micro/Zero
13   #endif
14   Serial.begin(115200);
15   //Initialize the PN532 breakout board
16   cie.begin();
17 \}
\end{DoxyCode}
 Then, in your loop, wait for a card then read its I\+D\+\_\+\+Servizi (a low-\/security unique identifier)


\begin{DoxyCode}
1 \{C++\}
2 void loop(void) \{
3   //Let's see if a card is present
4   bool cardDetected = cie.detectCard();
5   if (!cardDetected) \{
6     //No card present, we wait for one
7     delay(100);
8     return;
9   \}
10 
11   //Good! A card is present, let's read the ID!
12   word bufferLength = EF\_ID\_SERVIZI\_LENGTH;
13   byte buffer[EF\_ID\_SERVIZI\_LENGTH];
14 
15   if (!cie.read\_EF\_ID\_Servizi(buffer, &bufferLength)) \{
16     Serial.print(F("Error reading EF.ID\_SERVIZI"));
17     delay(1000);
18     return;
19   \}
20 
21   //We were able to read the ID\_Servizi, print it out!
22   Serial.print(F("EF.ID\_Servizi: "));
23   cie.printHex(buffer, bufferLength);
24 \}
\end{DoxyCode}


\subsection*{More examples}

This library comes with an {\itshape examples} directory. You can load and run examples from the Arduino I\+DE by clicking the File menu -\/$>$ Examples -\/$>$ cie 532.

\subsection*{Useful links}


\begin{DoxyItemize}
\item The C\+IE 3.\+0 chip specification (italian) \href{http://www.agid.gov.it/sites/default/files/documentazione/cie_3.0_-_specifiche_chip.pdf}{\tt http\+://www.\+agid.\+gov.\+it/sites/default/files/documentazione/cie\+\_\+3.\+0\+\_\+-\/\+\_\+specifiche\+\_\+chip.\+pdf}
\item Technical specification for the European Card for e-\/\+Services and National e-\/\+ID Applications \href{http://www.unsads.com/specs/IASECC/IAS_ECC_v1.0.1_UK.pdf}{\tt http\+://www.\+unsads.\+com/specs/\+I\+A\+S\+E\+C\+C/\+I\+A\+S\+\_\+\+E\+C\+C\+\_\+v1.\+0.\+1\+\_\+\+U\+K.\+pdf}
\item Specifiche tecniche del documento digitale unificato (italian) \href{http://www.agid.gov.it/sites/default/files/leggi_decreti_direttive/specifiche_tecniche_del_documento_digitale_unificato_v.1.0.0.pdf}{\tt http\+://www.\+agid.\+gov.\+it/sites/default/files/leggi\+\_\+decreti\+\_\+direttive/specifiche\+\_\+tecniche\+\_\+del\+\_\+documento\+\_\+digitale\+\_\+unificato\+\_\+v.\+1.\+0.\+0.\+pdf} 
\end{DoxyItemize}